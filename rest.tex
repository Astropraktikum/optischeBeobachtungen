Es gilt für die scheinbare Helligkeit und bei konstantem Abstand: 
\begin{equation}
m_2 - m_1 = 2.5\cdot \log(\frac{L_1}{L_2}. 
\end{equation}, 
wobei $m_1, m_2$ die scheinbaren Helligkeiten und $L_1. L_2$  die Leuchtkräfte der beiden Sterne im gleichem Abstand zum Beobachter sind.

Durch Umformung ergibt sich: 
\begin{equation}
\frac{L_1}{L_2} = 10^{2.5 \cdot (m_2 - m_1)}. 
\end{equation}

Für ein Doppelstern mit den scheinbaren Helligkeiten m1 und m2 ergibt sich also: 
Die Gesamtleuchtkraft ergibt sich durch Addition der einzelnen Leuchtkräfte und die Gesamtmagnitude nach Umrechnung der Gesamthelligkeit. 

\begin{equation}
L_{ges} = L_1 + L_2 = L_1 \cdot (1 + \frac{L_2}{L_1}) = L_1 \cdot (1 + 10^{2.5 \cdot (m_1 - m_2)}). 
\end{equation}

Für die Gesamtmagnitude ergibt sich:
\begin{equation}
m_{ges} = m_1 - 2.5\cdot \log(\frac{L_{ges}}{L_1}) = m_1 - 2.5\cdot \log(1 + 10^{2.5 \cdot (m_1 - m_2)}).
\end{equation}