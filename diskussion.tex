\section{Diskussion}
Bei sowohl den Aufnahmen mit dem Handteleskop als auch mit den großen Teleskopen in den Kuppeln wurden die Einschränkungen bei der Auflösung bedingt durch das Seeing deutlich. Dieser Effekt kann theoretisch durch die Beobachtung eines als annähernd punktförmig wahrgenommenen Sternes ausgeglichen werden: Man errechnet aus dem beobachteten Flimmern des Sterns eine Korrektur, die dann auf die eigentliche Beobachtung angwendet wird. Da dies in Echtzeit geschehen muss, ist dieses Verfahren allerdings sehr aufwendig und steht hier nicht zur Verfügung. 

\subsection{Beobachtung mit dem Teleskop im Garten}
Der Grund, weshalb nur drei Monde beobachtbar sind